\documentclass[aspectratio=169,table,xcdraw,18pt,portugues]{beamer}

\input{wsconfig_beamer.tex}

% =====================================================================

\title{Apresentação de Exemplo}
\subtitle{Testando a funcionalidade da ferramenta}
\author{\texorpdfstring{Wesley Rodrigues \textit{wesley.my@email.com}}{}}
\institute{
    Mestrado Profissional em Computação Aplicada\\
    \small{Instituto de Pesquisas Tecnológicas do Estado de São Paulo}\\
    
    }
\date{\small{November 2022}}

% =====================================================================

\begin{document}

% ---------------------------------------------------------------------

{
\setbeamertemplate{footline}{}
\frame{
    \hfill\includegraphics[width=0.35\textwidth]{logos/logo_ipt_light_700_350.png}
    \titlepage
}
}

% ---------------------------------------------------------------------

\begin{frame}
    \frametitle{Sample frame title}
    This is some text in the first frame.\\
    This is some text in the first frame.

    And this is some more text in the first frame.
\end{frame}

% ---------------------------------------------------------------------

\begin{frame}[fragile]
    \frametitle{Some code}

    \begin{verbatim}
        #include <stdio.h>

        int main() {
            printf("Hello World!");
            return 0;
        }
    \end{verbatim}
\end{frame}

% ---------------------------------------------------------------------

\begin{frame}
    \frametitle{Amazing Equations!}
    \[
        \lim_{x\to 0}{\frac{e^x-1}{2x}}
        \overset{\left[\frac{0}{0}\right]}{\underset{\mathrm{H}}{=}}
        \lim_{x\to 0}{\frac{e^x}{2}}={\frac{1}{2}}
    \]

    \begin{align*}
        f(x)  & = a x^2+b x +c & g(x)  & = d x^3   \\
        f'(x) & = 2 a x +b     & g'(x) & = 3 d x^2
    \end{align*}

    \[f(x) = \left\{
        \begin{array}{lr}
            x^2 & : x < 0   \\
            x^3 & : x \ge 0
        \end{array}
        \right.
    \]
\end{frame}

% ---------------------------------------------------------------------

{
\setbeamertemplate{footline}{}
\setbeamercolor{frametitle}{fg=bgcolor}
\usebackgroundtemplate{
    \includegraphics[width=\paperwidth]{backgrounds/img1.jpg}
}
\begin{frame}
    \frametitle{This have an bg image}
    Hello from Latex!
\end{frame}
}

% ---------------------------------------------------------------------

\begin{frame}
    \frametitle{This does not =D}
    Hello from Latex!
\end{frame}

% ---------------------------------------------------------------------

\begin{frame}
    \frametitle{A regular image}
    \centering
    \includegraphics[width=\textwidth,height=0.6\textheight,keepaspectratio]{backgrounds/img2.jpg}
\end{frame}

% ---------------------------------------------------------------------

\begin{frame}
    \frametitle{Images in Two-column slide}
    \begin{columns}[t]
        \column{0.5\textwidth}
        \centering
        \Large{A Nice Image}\newline\newline
        \includegraphics[width=\textwidth,height=0.5\textheight,keepaspectratio]{backgrounds/img1.jpg}
        \column{0.5\textwidth}
        \centering
        \Large{Another Image}\newline\newline
        \includegraphics[width=\textwidth,height=0.5\textheight,keepaspectratio]{backgrounds/img2.jpg}
    \end{columns}
\end{frame}

% ---------------------------------------------------------------------

\begin{frame}
    \frametitle{Images + Text in Two-column slide}
    \begin{columns}[t]
        \column{0.5\textwidth}
        \centering
        \Large{A Nice Image}\newline\newline
        \includegraphics[width=\textwidth,height=0.5\textheight,keepaspectratio]{backgrounds/img2.jpg}
        \column{0.5\textwidth}
        \centering
        \Large{Super Text}\newline\newline
        \small{\lipsum[1][1-3]}
    \end{columns}
\end{frame}

% ---------------------------------------------------------------------

\begin{frame}
    \frametitle{Citations}

    If you want to cite some amazing book \cite{Genius1801}, or maybe
    an incredible article \cite{LaTeX2020} from the web, or even an
    amazing author \cite{SomeAuthor2022}, you can! But remember, this is
    not an article, this is a presentation.

    Acronyms like \ac{i3e} or \ac{smart} are also available.

\end{frame}

% ---------------------------------------------------------------------

\begin{frame}[t,allowframebreaks]
    \frametitle{References}
    \printbibliography
\end{frame}

% ---------------------------------------------------------------------

\begin{frame}
    \frametitle{Acronyms}
    \begin{acronym}[ICANN]
        \input{acronyms.tex}  % <== This loads a separate list of acronyms
    \end{acronym}
\end{frame}

% ---------------------------------------------------------------------

{
\setbeamercolor{background canvas}{bg=fgcolor}
\setbeamertemplate{footline}{}
\begin{frame}
    \centering
    {\vskip 5em}
    {\color{bgcolor} \fontsize{30pt}{35pt}\selectfont Thank You!}\\
    {\vskip 3em}
    {\color{c3color} \fontsize{20pt}{25pt}\selectfont Wesley Rodrigues}\\
    {\vskip 1em}
    {\color{c3color} \fontsize{14pt}{16pt}\selectfont wesley@mymail.com}\\
    {\vskip 2em}
    \includegraphics[width=0.35\textwidth]{logos/logo_ipt_dark_700_350.png}
\end{frame}
}

% ---------------------------------------------------------------------

\end{document}
